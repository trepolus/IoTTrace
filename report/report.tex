\documentclass[11pt,conference,a4paper,onecolumn,romanappendices]{IEEEtran}
\usepackage[utf8]{inputenc}
\usepackage[english]{babel}
\author{Lucas Colantuono, Lucas Kummer, Shanan Lynch, Samuel Sedlmeir}

\title{Improving local transports by analysing taxi meta data}
\date{\today}
\markboth{Improving local transports by analysing taxi meta data}{}

\author{\IEEEauthorblockN{Lucas Colantuono}
\IEEEauthorblockA{INSA Lyon \\
lucas.colantuono@insa-lyon.fr}
\and
\IEEEauthorblockN{Lucas Kummer}
\IEEEauthorblockA{INSA Lyon\\
lucas.kummer@insa-lyon.fr}
\and
\IEEEauthorblockN{Shanan Lynch}
\IEEEauthorblockA{INSA Lyon\\
shanan.lynch@insa-lyon.fr}
\and
\IEEEauthorblockN{Samuel Sedlmeir}
\IEEEauthorblockA{INSA Lyon\\
S.Sedlmeir@campus.lmu.de}}

\begin{document}

\maketitle

\tableofcontents
\newpage

\begin{abstract}
 
\end{abstract}

\section{Introduction}
\label{sec:Introduction}
All over the world big cities are struggling to fight against air pollution as a result of individual transport. This does not just concern developing countries or emerging countries, but also developed first world countries: Beijing will ban certain cars to prevent an overriding of the emission limits \cite{beij}, whereas the city of Oxford is planning to establish a petrol car free zone \cite{oxfo}. \\
While bans forbid people to use their cars, a more sustainable way would be to make people use the local transport voluntarily by improving it where necessary. Thus we need a reliable possibility to learn about people's possible needs. As a lot of people choose a taxi where there is a unsatisfying offer of transport, we could use taxi meta data in order to analyse automatically possible useful routes for new bus lines for example. \\
To achieve this goal, we use data collected by taxis in Shanghai in 2007. These data contain GPS positions, a timestamp, the status, the direction in which the taxi is moving as well as a Taxi ID. In order to analyse this data we set up a InfluxDB Database and a Chronograf and Grafana Web Server, where the data is being inserted to by using a Python script.
\section{Related work}
There are several different articles, that analyse the same dataset.
\section{Problem formulation}
The main problem is to find routes, that a bigger number of taxi trips have in common. Therefore we could concentrate on taxi routes with a close place of departure and/or a close destination.
\section{Analysis}
\section{Results}
\section{Summary}

\newpage
\bibliography{biblio}
\bibliographystyle{IEEEtran}

\end{document}